Attempts to relate the electronic structure to plasticity have been made, but even recently such studies have tended to find structural, physical and elastic properties of ``complex'' materials and then infer the relative ductility on the basis of these properties, usually the ductility criteria discussed above. For example some ternary metal nitrides with chemistry obeying \ce{Ti_xM_{1-x}N} are of interest as wear resistant coatings and increased toughness is very desirable and studies have predicted that the use of \ce{Mo} or \ce{W} as alloying additions significantly improve the toughening, indeed the effect has been dubbed ``\emph{supertoughening}'' \cite{Sangiovanni2010,Sangiovanni2011}. The studies use \emph{ab initio} calculations to find the elastic response of the alloyed crystal to an applied strain. The study finds a number of interesting things including the development of a layered electronic structure and trends for elastic properties, particularly the Pugh ratio, $B/G$, with the valence electron concentration. The authors speculate about selective local responses to stress, though with no further exploration since the elastic constants were calculated from the energy changes under various applied \emph{uniform} shear strains.

However the conclusions drawn from these elastic simulations about plasticity can be, at best, qualitative. In fact these studies employ on arguments of easily broken bonds since ``during dislocation motion bonds are broken and reformed and, obviously, dislocation glide will occur more easily in planes normal to those containing weaker bonds'' \cite{Sangiovanni2011} without further justification. Clearly this is not a mechanistic explanation for the effect of chemical bonding on plasticity relying as it does on dated and empirical ratios of elastic constants.


Other studies recognise the limits of simple ductility criteria and use the concept of Peierls stress more directly \cite{Music2008,Emmerlich2009,Gouriet2015}, but use the established results for simple materials in which no accounting for local heterogeneities is made, distribution of strains is always uniform and often the materials are simply taken to be isotropic continua. The conclusions that can be drawn from applying a simple model to a complex structure are limited to those that could be drawn from the simple model: i.e. that if the elastic constants and stacking fault energies take suitable values then the dislocation will be wider or that the ratio of the slip plane spacing and the Burgers vector will allow easier slip. Given these are evident from most of the formulations for the Peierls stress it does not shed much light on the ideal material to use.


