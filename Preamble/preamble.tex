\usepackage[inner=3cm,outer=3cm,top=2.5cm,bottom=2.5cm]{geometry}

\usepackage{booktabs}



\usepackage[T1]{fontenc}

\usepackage{setspace}
\usepackage{helvet}
\usepackage{url}
\usepackage[version=4]{mhchem}
\usepackage{afterpage}


\usepackage[hidelinks]{hyperref}
\hypersetup{final=true}



\usepackage[font={sf,scriptsize},labelfont=bf]{caption}
\captionsetup[subfigure]{width=\textwidth,font={sf,scriptsize},labelfont=bf}




\usepackage{subcaption}
\usepackage{graphicx}
\usepackage[section]{placeins}
\usepackage[version=4]{mhchem}






\usepackage{booktabs} 
\usepackage{multirow}



\usepackage{amsfonts}
\usepackage{mathtools}
\usepackage{amssymb}
\usepackage[detect-all]{siunitx}
\usepackage{bm}

     
\newcommand{\dd}{\mathop{}\,\mathrm{d}}
\newcommand{\oldepsilon}{\epsilon}
\renewcommand{\epsilon}{\varepsilon}






\usepackage{epigraph}




\usepackage{pdfpages}

\setcounter{secnumdepth}{2}



\usepackage[backend=biber,
date=year,
eprint=false,
natbib=true,
giveninits=true,
citestyle=numeric-comp,
bibstyle=ieee,
sortcites=true,
sorting=none,
maxbibnames=6,
minbibnames=6]{biblatex}
\addbibresource{library.bib}
\addbibresource{websites.bib}

\AtEveryBibitem{\ifentrytype{article}{\clearfield{issn}}{}}
\AtEveryCitekey{\ifentrytype{article}{\clearfield{issn}}{}}
\AtEveryBibitem{\ifentrytype{article}{\clearfield{url}}{}}
\AtEveryCitekey{\ifentrytype{article}{\clearfield{url}}{}}
\AtEveryBibitem{\ifentrytype{article}{\clearfield{month}}{}}
\AtEveryBibitem{\ifentrytype{article}{\clearfield{day}}{}}
% changes the default name `Bibliography` -> `References'
\AtEveryBibitem{\ifentrytype{misc}{\clearfield{month}}{}}
\renewcommand{\bibname}{References}


\DefineBibliographyStrings{english}{%
  bibliography = {References},
}

\usepackage{filecontents}



\usepackage{fancyhdr}
\usepackage{flipbook}



\pagestyle{fancy}
\fancyhf{}

\fancyhead[lo]{\slshape\nouppercase{\rightmark}}
\fancyhead[re]{\slshape\nouppercase{\leftmark}}
\fancyhead[ro,le]{\thepage}

% \rfoot[~ \\
% \fbImageF{./flipbook/disloc}{png}{scale=0.3}]{}


% \rfoot[]{
% \setlength\unitlength{1cm}
%   \begin{picture}(0,0)
%     \put(2.5,-2.5){
%     \fbImageB{/flipbook/disloc}{png}{scale=0.25}
%     }
%   \end{picture}
% }








\usepackage{rotating}
\usepackage{pdflscape}


\usepackage{dcolumn}
\newcolumntype{d}[1]{D{.}{\cdot}{#1} }
\setlength{\headheight}{14.5pt}
\usepackage{bookmark}
\usepackage{csvsimple}
\usepackage{longtable}
\usepackage{booktabs}
\usepackage{listings}
\usepackage{color}
\definecolor{mygray}{rgb}{0.5,0.5,0.5}
\definecolor{mymauve}{rgb}{0.58,0,0.82}
\definecolor{myred}{rgb}{0.8,0.0,0.0}
\definecolor{mygreen}{rgb}{0.0, 0.28, 0.15}

\lstset{ %
  backgroundcolor=\color{white},   % choose the background color; you must add \usepackage{color} or \usepackage{xcolor}; should come as last argument
  basicstyle=\footnotesize\ttfamily,        % the size of the fonts that are used for the code
  breakatwhitespace=false,         % sets if automatic breaks should only happen at whitespace
  breaklines=false,                 % sets automatic line breaking
  captionpos=b,                    % sets the caption-position to bottom
  commentstyle=\color{mymauve},    % comment style
  deletekeywords={...},            % if you want to delete keywords from the given language
  escapeinside={\%*}{*)},          % if you want to add LaTeX within your code
  extendedchars=true,              % lets you use non-ASCII characters; for 8-bits encodings only, does not work with UTF-8
  keepspaces=true,                 % keeps spaces in text, useful for keeping indentation of code (possibly needs columns=flexible)
  keywordstyle=\color{blue},       % keyword style
  language=Octave,                 % the language of the code
  morekeywords={dict,len, def, elif, from, import, as, try, except, with}, % if you want to add more keywords to the set
  deletekeywords={system,grid, range},
  numbers=left,                    % where to put the line-numbers; possible values are (none, left, right)
  numbersep=5pt,                   % how far the line-numbers are from the code
  numberstyle=\tiny\color{mygray}, % the style that is used for the line-numbers
  rulecolor=\color{black},         % if not set, the frame-color may be changed on line-breaks within not-black text (e.g. comments (green here))
  showspaces=false,                % show spaces everywhere adding particular underscores; it overrides 'showstringspaces'
  showstringspaces=false,          % underline spaces within strings only
  showtabs=false,                  % show tabs within strings adding particular underscores
  stepnumber=4,                    % the step between two line-numbers. If it's 1, each line will be numbered
  stringstyle=\color{myred},     % string literal style
  tabsize=2,	                   % sets default tabsize to 2 spaces
}


\usepackage{datetime}

\newdateformat{monthyeardate}{%
  \monthname[\THEMONTH], \THEYEAR}

\usepackage{flipbook}




















































