\chapter{Conclusions and future work}



In this work the plastic behaviour of crystals has been investigated by use of the Peierls model. 

A new Peierls model has been created using python and the associated projects, Numpy and Scipy, that allows the creation of a two or three dimensional Peierls model rather than the previous one dimension used in previous models.

The linear elasticity based calculations are in broad agreement with experimental observations though quantitative results are unlikely to be accurate. The model does explain observations of softening in cementite as hydrogen is added via the changes in the single crystal elastic constants as calculated by density functional theory.

Dislocations in ionic solids were investigated, firstly with an original implementation of the electrostatic and Lennard-Jones energy calculations. This implementation was deliberately simple to avoid the introduction of artefacts or complications associated adjustments to improve the performance to atomic potentials, such as cut off distances. This was to ensure the results were as interpretable as possible.

It was shown for the <1\,1\,0>\{0\,0\,1\} slip system in sodium chloride that the short range repulsion energy, here modelled by the Lennard-Jones potential, and the electrostatic energy were sufficient to explain the observed Peierls stress. As the dislocation moved the energy components were shown to vary out of phase, in a parallel of the strain and misalignment energy components of previous Peierls models.

The <1\,1\,0>\{1\,\={1}\,0\} slip system could not be adequately described by the model, showing much larger energy changes than expected and physically implausible variations in the dislocation width. The size of the simulation and the effects of boundaries were shown not to be responsible for these observations by the use of LAMMPS routines to calculate the energy using more efficient implementations of the interatomic potentials. There is clearly an important factor in the behaviour of dislocations on this, the softer slip system, that are not accounted for in the model presented here; possible factors include core reconstruction or the lack of polarisability of the atomic potentials.

The complex crystal structures of the MAX phases were considered. These phases have clear chemical heterogeneity within the unit cell. Density functional theory calculations were used to calculate the stiffness of each of the regions separately. These results were in agreement with the macroscopic properties when combined with an equal stress assumption. The properties of these layers were shown to vary with the chemical nature of the layers, as characterised by the electronegativity, where the macroscopic properties did not.

These elastic properties, and the calculated generalised stacking fault energy were used in an adapted Peierls model to predict the Peierls stress of the MAX phases. The calculated Peierls stresses are in good agreement with the observed easy flow in the MAX phases, which is not adequately explained by the macroscopic properties alone. The effect was shown to be strong, with $^{\tau_p}\!/_{G}$ varying by five orders of magnitude as the electronegativity difference between the layers of the structure vary by \SI{1.566}{\electronvolt} from \SIrange{1.5880}{0.0214}{\electronvolt}

The shows that the local elastic heterogeneity, induced by a local chemical heterogeneity, on the scale of the dislocation core, can offer a route to tailoring the flow stress of complex crystals.

The effect was then investigated in the \ce{Ti2Ni} structure, which is a complex metallic alloy with a large unit cell containing 96 atoms. The slip system was investigated by in-situ micropillar compression, and slip was shown to occur on the \{1\,1\,1\} planes. The slip direction was hard to determine conclusively but the partial <2\,\={1}\,\={1}\,> is more likely than the full <1\,\={1}\,0>. A series of alloys were chosen to alter the chemical heterogeneity of the  \{1\,1\,1\} planes. The flow stress of the alloys was tested by nanoindentation of the \{1\,1\,1\} face of single crystals to ensure the Schmid factor remained constant. The hardness of the alloys varied significantly even though the range of heterogeneity was far smaller than that of the MAX phases; a reduction of \SI{\sim2}{\giga\pascal} from \SI{12.5}{\giga\pascal} to \SI{10.6}{\giga\pascal} or about \SI{15}{\percent} was observed for a change in the electronegativity difference of just \SI{0.066}{\electronvolt}.



There are two clear avenues for future work, an experimental route and a modelling route.

Experimentally the best validation of the model would be to confirm the Peierls stresses at cryogenic temperatures. The MAX phase results presented here agree with those experimental results that exist, i.e. that the flow stress of \ce{Ti2SiC3} at room temperature is around \SI{50}{\mega\pascal} which is similar to soft metals. However no direct evidence of the Peierls stress (the flow stress at \SI{0}{\kelvin}) exists.

Another obvious route for experimental work to build on this work, which assumes that the conclusions drawn here are valid, is to attempt to find a system that might be more industrially relevant. The \ce{Ti2Ni} results are important because they show the effect occurs in an cubic structure with enough slip systems that full plasticity is possible, unlike in the hexagonal MAX phases. However the reduction of the hardness from over \SI{12}{\giga\pascal} to over \SI{10}{\giga\pascal} is clearly insufficient to produce a alloy with enough ductility to be useful as a structural material.

Instead the best opportunities will be in phases that are close to being sufficiently ductile, allowing the altered structure to more easily reach that critical point at which the alloy is tough enough for use as a structural material. In particular aluminium bearing intermetallics, such as tantalum aluminides which have cubic crystal structures, would have sufficient corrosion resistance for higher temperature applications should the mechanical properties be tailored appropriately.

The modelling aspect of the project could be taken forward in a number of ways. The initial creation of the atomic arrays could be improved. Screw dislocation are an obvious avenue. Their behaviour can be quite different to edge dislocations.
































































































































