
\chapter{Dislocations in MAX phases}

\label{chap:dislocations_in_max_phases}
\graphicspath{{plastic_deformation/Figs/}}






In \autoref{chap:hetero_max_phases} elastic heterogeneity of the unit cell was discussed and the local elastic properties were shown to have a strong dependence on the chemical environment. As discussed in  \autoref{chap:plastic_deformation} and shown in \autoref{chap:peierls_model} the dislocation properties are very sensitive to the elastic properties of a crystal. This leads to the question: what is the effect of elastic heterogeneity on the dislocations in the MAX phases?

The elastic properties of the different layers of the MAX phases are not in the form of full elastic tensors so the approach taken in \autoref{chap:peierls_model} to use the full strain state cannot be applied. Instead a simplified model, adapted from that presented by \citet{Clegg2006} is used that relies on only simple strains and elastic constants. This model was shown to be in reasonable agreement with experiment across orders of magnitude of the Peierls stress and should be sufficient to show what the effect of accounting for elastic heterogeneity is.

The model